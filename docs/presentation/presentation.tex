\documentclass[aspectratio=169]{beamer}

% Build from repo root:
%   python3 scripts/extract_notebook_images.py --clean
%   pdflatex docs/presentation/presentation.tex
% Or from docs/presentation:
%   pdflatex presentation.tex

\usetheme{metropolis}
\setbeamertemplate{navigation symbols}{}
\setbeamertemplate{footline}[frame number]

\usepackage[utf8]{inputenc}
\usepackage{graphicx}
\usepackage{booktabs}
\usepackage{amsmath}

\graphicspath{
  {../figures/}
  {../figures/visualization_presentation/}
  {../paper/figures/}
  {docs/figures/}
  {docs/figures/visualization_presentation/}
  {docs/paper/figures/}
}

\title{Physics-Informed Map-Conditioned UE Localization}
\subtitle{Coarse-to-fine Transformer posterior with differentiable radio-map physics}
\author{Nir Tzur}
\date{\today}

\begin{document}

\begin{frame}
  \titlepage
\end{frame}

\begin{frame}{Story arc}
  \begin{itemize}
    \item Problem framing: localization is multi-modal and physics-constrained.
    \item Pipeline: OSM to scenes to radio maps to training and inference.
    \item Model: dual encoder, cross-attention fusion, coarse-to-fine posterior.
    \item Evidence: qualitative results, error statistics, and training dynamics.
  \end{itemize}
\end{frame}

\begin{frame}{Problem and goals}
  \begin{itemize}
    \item Input: sparse, irregular radio measurements across multiple protocol layers.
    \item Challenge: inverse mapping is ill-posed and multi-modal in urban scenes.
    \item Goal: map-conditioned probabilistic posterior with calibrated uncertainty.
    \item Constraint: predictions must remain consistent with propagation physics.
  \end{itemize}
\end{frame}

\begin{frame}{End-to-end pipeline}
  \begin{columns}[T,onlytextwidth]
    \begin{column}{0.45\textwidth}
      \begin{itemize}
        \item OSM and GIS data define scene geometry and semantics.
        \item Differentiable ray tracing yields radio maps and measurements.
        \item Train a transformer to amortize inference over UE position.
        \item Optional refinement improves low-confidence predictions.
      \end{itemize}
    \end{column}
    \begin{column}{0.55\textwidth}
      \centering
      \includegraphics[width=\linewidth]{viz_cell12_img01.png}
      \vspace{0.2em}
      {\footnotesize Scene and radio map from Sionna RT.}
    \end{column}
  \end{columns}
\end{frame}

\begin{frame}{Map context for localization}
  \centering
  \includegraphics[width=0.95\linewidth]{environment_context.png}
  \vspace{0.2em}
  {\footnotesize Radio propagation map and semantic building context.}
\end{frame}

\begin{frame}{Inputs and map channels}
  \begin{itemize}
    \item Radio features: RT, PHY/FAPI, and MAC/RRC measurements.
    \item Map features: radio channels (path gain, ToA, SNR, SINR, throughput)
      and OSM channels (height, material, footprint, road, terrain).
  \end{itemize}
  \vspace{0.4em}
  \centering
  \includegraphics[width=0.98\linewidth]{viz_cell09_img00.png}
\end{frame}

\begin{frame}{Model architecture}
  \centering
  \includegraphics[width=\linewidth]{architecture_diagram.pdf}
  \vspace{0.2em}
  {\footnotesize Dual encoders with cross-attention and coarse-to-fine heads.}
\end{frame}

\begin{frame}{Coarse-to-fine posterior}
  \begin{columns}[T,onlytextwidth]
    \begin{column}{0.45\textwidth}
      \begin{itemize}
        \item Coarse grid predicts a heatmap over spatial cells.
        \item Top-K cells refine to Gaussian components.
        \item Mixture posterior encodes multi-modality and uncertainty.
      \end{itemize}
      \vspace{0.5em}
      \[
        p(\mathbf{y}) = \sum_{k=1}^{K} \pi_k \mathcal{N}(\mathbf{y}; \mu_k, \Sigma_k)
      \]
    \end{column}
    \begin{column}{0.55\textwidth}
      \centering
      \includegraphics[width=0.85\linewidth]{posterior_density.png}
      \vspace{0.2em}
      {\footnotesize Example posterior density.}
    \end{column}
  \end{columns}
\end{frame}

\begin{frame}{Training objective and physics loss}
  \begin{block}{Total objective}
    \[
      \mathcal{L} =
      \mathcal{L}_{\text{coarse}} +
      \lambda_f \mathcal{L}_{\text{fine}} +
      \lambda_p \mathcal{L}_{\text{phys}}
    \]
  \end{block}
  \vspace{0.3em}
  \begin{itemize}
    \item Coarse: cross-entropy over grid cells.
    \item Fine: Gaussian NLL with heteroscedastic uncertainty.
    \item Physics: differentiable lookup of radio maps via bilinear sampling.
  \end{itemize}
  \vspace{0.3em}
  \[
    \mathcal{L}_{\text{phys}} =
    \sum_f w_f \| m_f^{\text{obs}} - R_f(\hat{\mathbf{y}}) \|^2
  \]
\end{frame}

\begin{frame}{Inference and refinement}
  \begin{columns}[T,onlytextwidth]
    \begin{column}{0.45\textwidth}
      \begin{itemize}
        \item Predict heatmap and top-K candidate cells.
        \item Refine to continuous positions with uncertainty.
        \item Optional MAP refinement combines network posterior
          with radio-map likelihood.
      \end{itemize}
    \end{column}
    \begin{column}{0.55\textwidth}
      \centering
      \includegraphics[width=\linewidth]{viz_cell15_img02.png}
      \vspace{0.2em}
      {\footnotesize Prediction vs ground truth with uncertainty.}
    \end{column}
  \end{columns}
\end{frame}

\begin{frame}{Evaluation metrics}
  \centering
  \includegraphics[width=0.95\linewidth]{viz_cell20_img03.png}
  \vspace{0.2em}
  {\footnotesize Error histogram, CDF, and box plot for batch evaluation.}
\end{frame}

\begin{frame}{Training dynamics}
  \centering
  \includegraphics[width=0.9\linewidth]{training_dynamics.png}
  \vspace{0.2em}
  {\footnotesize Loss curves and median localization error over steps.}
\end{frame}

\begin{frame}{Project status and integration}
  \begin{columns}[T,onlytextwidth]
    \begin{column}{0.58\textwidth}
      \begin{itemize}
        \item Scene generation from OSM.
        \item Differentiable ray tracing and dataset generation.
        \item Dual-encoder transformer with fusion.
        \item Training and evaluation pipeline.
        \item Streamlit web demo.
      \end{itemize}
    \end{column}
    \begin{column}{0.42\textwidth}
      \centering
      \begin{tabular}{@{}ll@{}}
        \toprule
        Milestone & Status \\
        \midrule
        Scene generation & Complete \\
        Data generation & Complete \\
        Model & Complete \\
        Training & Complete \\
        Web UI & Complete \\
        \bottomrule
      \end{tabular}
    \end{column}
  \end{columns}
\end{frame}

\begin{frame}{Conclusion and next steps}
  \begin{itemize}
    \item Map-conditioned transformer yields a calibrated, multi-modal posterior.
    \item Physics-informed loss aligns predictions with radio propagation.
    \item Pipeline is end-to-end: OSM to scenes to training to inference.
    \item Next: larger-scale city runs, ablations, and uncertainty calibration.
  \end{itemize}
\end{frame}

\end{document}
